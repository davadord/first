\begin{abstract}
El presente Proyecto de Grado tiene el prop\'osito de implementar una Aplicaci\'on Web, utilizando conjuntamente las metodolog\'ias \textbf{MDA} (Model Driven Architecture) y \textbf{MERODE} (Model - driven, Existence - dependency Relation Object - oriented DEvelopment), para facilitar la Administraci\'on de un Evento en un Grupo de Investigaci\'on.

Es importante mencionar que anteriormente se ha implementado, como Proyecto de Tesis, un producto llamado AppVlir8\footnote{\url{http://www.vlir8.espol.edu.ec}}, el mismo que es tambi\'en un Portal Web de Administraci\'on de eventos. Nuestra idea es retomar el dise\~no original de este software y modificarlo, corrigiendo defectos y agregando mejoras cuya necesidad se ha identificado durante el tiempo en que el software ha estado en producci\'on (a partir de Septiembre del 2006).

En el primer cap\'itulo se se\~nalan los antecedentes de retomar este Proyecto, el hecho de que MERODE si ayuda en el dise\~no independiente del dominio, vital para seguir la metodolog\'ia MDA, aunque se va a usar la misma tecnolog\'ia J2EE, utilizada en AppVlir8, pero con la especificaci\'on JSP 2.0 siguiendo el patr\'on de dise\~no MVC 2, los objetivos que se espera alcanzar y la justificaci\'on del desarrollo de los m\'odulos.

En el segundo cap\'itulo se redactan los fundamentos te\'oricos en los que se basa este Proyecto de Grado, justificando el uso de cada elemento incluido en la arquitectura del mismo.

En el tercer cap\'itulo se documentan los requerimientos funcionales y no funcionales levantados para mejorar el dise\~no del anterior Sistema. Adem\'as de la especificaci\'on del PIM y del PSM, previos a la fase de la transformaci\'on al c\'odigo.

En el cuarto cap\'itulo se mencionan los cambios realizados en la Arquitectura del Sistema anterior y las novedades en el actual, para poder satisfacer algunos requerimientos claves del usuario del Sistema.

En el quinto cap\'itulo se realiza un an\'alisis con respecto a la implementaci\'on realizada y las pruebas que se realizaron para la entrega de un producto de calidad.

Finalmente se exponen las conclusiones y recomendaciones del Proyecto de Grado.
\end{abstract}
